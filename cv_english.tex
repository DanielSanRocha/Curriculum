%!TEX program = xelatex

%%%%%%%%%%%%%%%%%%%%%%%%%%%%%%%%%%%%%%%%%
% Stylish Curriculum Vitae
% LaTeX Template
% Version 1.0 (18/7/12)
%
% This template has been downloaded from:
% http://www.LaTeXTemplates.com
%
% Original author:
% Stefano (http://stefano.italians.nl/)
%
% IMPORTANT: THIS TEMPLATE NEEDS TO BE COMPILED WITH XeLaTeX
%
% License:
% CC BY-NC-SA 3.0 (http://creativecommons.org/licenses/by-nc-sa/3.0/)
%
% The main font used in this template, Adobe Garamond Pro, does not
% come with Windows by default. You will need to download it in
% order to get an output as in the preview PDF. Otherwise, change this
% font to one that does come with Windows or comment out the font line
% to use the default LaTeX font.
%
%%%%%%%%%%%%%%%%%%%%%%%%%%%%%%%%%%%%%%%%%
\documentclass[a4paper, oneside, final]{report} % Paper options using the scrartcl class
\usepackage[margin=2.0cm]{geometry}
\usepackage{titlesec} % Allows creating custom \section's
\usepackage{marvosym} % Allows the use of symbols
\usepackage{tabularx,colortbl} % Advanced table configurations
\usepackage{fontspec} % Allows font customization

\defaultfontfeatures{Mapping=tex-text}
% \setmainfont{Adobe Garamond Pro} % Main document font

\titleformat{\section}{\large\scshape\raggedright}{}{0em}{}[\titlerule] % Section formatting

\addtolength{\voffset}{-1cm} % Adjust the vertical offset - less whitespace at the top of the page
\addtolength{\textheight}{1cm} % Adjust the text height - less whitespace at the bottom of the page

\newcommand{\gray}{\rowcolor[gray]{.90}} % Custom highlighting for the work experience and education sections

\begin{document}

\begin{center} 
{\addfontfeature{LetterSpace=20.0}\fontsize{36}{36}\selectfont\scshape Daniel Santana Rocha} % Your name at the top

\vspace{1.5cm} 

\end{center}
\section{Summary}

I have a undegraduate degree in Mathematics Applied with enphasis in Scientific Computation in UFRJ (Federal University of Rio de Janeiro - 2016) and master in mathematics by IMPA (Instituto Nacinal de Matemática Pura e Aplicada - 2015).
\\ \\
Has experience in Mathematics in the area of Algebraic Geometry, Number Theory and Combinatorics.
\\ \\
Expertise in Machine Learning and bigdata with enphasis in NLP (sklearn, gensim, tensorflow, scipy, numpy, pytorch, doc2vec, word2vec and several other technologies and frameworks) and Reinforcement Learning (Thomson sampling and Q Learning).
\\ \\
Expertise in the area of bigdata with Hadoop, Spark (pyspark, scala), Google Cloud (BigQuery, VertexAI), Databricks.
\\ \\
Advanced knowledge: Python, Scala, Rust for machine learning and bigdata infrastructure. Good knowledge of javascript (Vue, ReactJs) for web interfaces.
\\ \\
Expertise with databases like Cassandra, Redis, Elasticsearch, Mongo, Openlink Virtuoso (SPARQL), InfluxDB and Others
\\ \\
Worked with several Big Data technologies like Kafka, Spark.
\\ \\
Advanced knowledge in containerized deployments with docker + docker compose, kubernetes, docker-swarm, VMs and automation of deployments in these technologies.
\\ \\
Basic knowledge in devops for maintenance of Kubernetes and docker swarm.
\\ \\
Expertise in linux server administration.
\\ \\
Very advanced english (speak since childhood), basic knowledge of french.

%----------------------------------------------------------------------------------------
%	Sobre Mim
%----------------------------------------------------------------------------------------
\section{About Me}
I love Mathematics and Programming, almost the Master more young of Brazil, also work with programming since 15 years in ml/bigdata environment like electoral statistics.
Already worked for Globo (higher rank Machine Learning Engineer focused in NLP, MAB (AB with reinforcement learning) and automated deploy) and Ifood (Machine Learn Engineer/Data Scientist focused in calculating the time of arrival of delivery).


\section{Projects}


\begin{tabularx}{0.97\linewidth}{>{\raggedleft\scshape}p{2cm}X}
  \gray Employer & Personal Challenge\\
  
  \gray Title & \textbf{Snake Game with Q-Learning}\\
    & Technolgies
  Pytthon, pandas, tensorflow.

  https://github.com/DanielSanRocha/SnakeAI
  
  https://danielsanrocha.com/snakeai
  \\
\end{tabularx}

\begin{tabularx}{0.97\linewidth}{>{\raggedleft\scshape}p{2cm}X}
  \gray Employer & Personal Use\\
  \gray Title & \textbf{Xatu Observer - Monitor of APIs, Services, Docker Containers}\\
    & Technolgies
  Scala, telegram API, finagle, MySQL, Redis, ElasticSearch.

  Monitor and collect logs and send it to elasticsearch for analysis.

  https://github.com/danielsanrocha/xatu-observer  
  \\
\end{tabularx}

  
\vspace{12pt}

%----------------------------------------------------------------------------------------
%	WORK EXPERIENCE
%----------------------------------------------------------------------------------------
\begin{center}

\section{Professinal History}

\begin{tabularx}{0.97\linewidth}{>{\raggedleft\scshape}p{2cm}X}
\gray Period & \textbf{February 2025 --- Currently}\\
\gray Employer & \textbf{WL3i - Fundo de Investimentos} \hfill Rio de Janeiro, Brasil\\
\gray Title & \textbf{Mathematician and Software Developer}\\
Tech & Scala, Finatra, Finagle, python, pandas.\\
\gray About & Data mining and system development for finding and negotiating the acquisiton of "precatórios" rights.
\end{tabularx}
  
\vspace{12pt}

\begin{tabularx}{0.97\linewidth}{>{\raggedleft\scshape}p{2cm}X}
\gray Period & \textbf{July 2024 --- January 2025}\\
\gray Employer & \textbf{iFood} \hfill Sao Paulo, Brasil\\
\gray Title & \textbf{Machine Learn Engineer 2B}\\
Tech & Kafka, databricks, aws, rust, scala, python, scala, java, redis. \\
\end{tabularx}

\vspace{12pt}

\begin{tabularx}{0.97\linewidth}{>{\raggedleft\scshape}p{2cm}X}
\gray Period & \textbf{January 2019 --- February 2024}\\
\gray Employer & \textbf{Globo} \hfill Rio de Janeiro, Brasil\\
\gray Title & \textbf{Specialist Bigdata Machine Learning Engineer}\\
TECH & NLP, Bert, Kafka, thompson sampling, google cloud, sklearn, python, scala, hadoop, spark, java, redis, virtuoso, elasticsearch, kibana, docker, docker swarm, kubernetes, gensim, tensorflow, pytorch and other technologies. \\
\end{tabularx}

\vspace{12pt}

\begin{tabularx}{0.97\linewidth}{>{\raggedleft\scshape}p{2cm}X}
\gray Period & \textbf{Março 2016 --- Junho 2016}\\
\gray Employer & \textbf{Federal University of Rio de Janeiro - UFRJ} \hfill Rio de Janeiro, Brasil\\
\gray Title & \textbf{Professor}\\
Tech & Professor at UFRJ programming (Java, tomcat, Android, Angular, REST Server).
\end{tabularx}

\vspace{12pt}

\begin{tabularx}{0.97\linewidth}{>{\raggedleft\scshape}p{2cm}X}
\gray Period & \textbf{Março 2014 --- Junho 2015}\\
\gray Employer & \textbf{TECHNO Soluções} \hfill São João de Meriti/RJ, Brasil\\
\gray Title & \textbf{Trainee in Bigdata Analisis for Electoral Statistics}\\
 Tech & PHP, MySQL, Java
\end{tabularx}

%----------------------------------------------------------------------------------------
%	EDUCATION
%----------------------------------------------------------------------------------------

\section{Academic}

\begin{tabularx}{0.97\linewidth}{>{\raggedleft\scshape}p{2cm}X}
\gray Period & \textbf{2016 --- StandBy}\\
\gray Title & \textbf{PhD in Mathematics}\\
\gray Institute & \textbf{Instituto Nacional de Matemática Pura e Aplicada} \hfill Rio de Janeiro, Brasil \\
& 	Orientador: Eduardo Esteves \\
&	Bolsista: Conselho Nacional de Desenvolvimento Científico e Tecnológico \\
&	Título da Dissertação: Limit of Linear Series
\end{tabularx}

\vspace{12pt}

\section{Academic}

\begin{tabularx}{0.97\linewidth}{>{\raggedleft\scshape}p{2cm}X}
\gray Period & \textbf{2013 --- 2015}\\
\gray Title & \textbf{Master in Mathematics}\\
\gray Institute & \textbf{Instituto Nacional de Matemática Pura e Aplicada} \hfill Rio de Janeiro, Brasil \\
& 	Orientador: Eduardo Esteves \\
&	Co-orientador: Emanuel A.S. \\
&	Bolsista: Conselho Nacional de Desenvolvimento Científico e Tecnológico \\
&	Título da Dissertação: Limite de Pontos de Weierstrass, Ano de obtenção: 2015
\end{tabularx}

\vspace{12pt}

\begin{tabularx}{0.97\linewidth}{>{\raggedleft\scshape}p{2cm}X}
\gray Periodo & \textbf{Janeiro 2015 --- Março 2016}\\
\gray Título & \textbf{Graduação em Matemática Aplicada}\\
\gray Instituto & \textbf{University of Rio de Janeiro - UFRJ} \hfill Rio de Janeiro, Brasil \\
\gray Ênfase & \textbf{Cientific Computation}
\end{tabularx}

%----------------------------------------------------------------------------------------
%	SKILLS
%----------------------------------------------------------------------------------------

\section{Expertises}

\begin{tabular}{ @{} >{\bfseries}l @{\hspace{6ex}} l }
Programming Language & Python, Scala, Rust, SQL (BigQuery and MySQL/MariaDB), Javascript, Haskell \\
Protocols & XML, JSON, REST, TCP/IP, HTTP, HTTPS \\
Databases & Cassandra, SQL Databases, MongoDB, Redis, Elasticsearch \\
Tools & Git, GitLab, Latex, Nginx, Kafka \\
Frameworks  & Spark (scala and pyspark), pandas, scikit-learn, gensim, spacy, React, Express (NodeJS), Vue, sklearn \\
Others & NLP (Word2Vec, Doc2Vec, TFIDF, Bert), ML (Thompson sampling, Q learning) \\
Others &  Gaussian Process, Microeconomics
\end{tabular}

%----------------------------------------------------------------------------------------

%----------------------------------------------------------------------------------------
%	Languages
%----------------------------------------------------------------------------------------

\section{Languages}

\begin{tabular}{ @{} >{\bfseries}l @{\hspace{6ex}} l }
Português & Nativo.\\
English & Advanced.\\
French & Understand little, Speak little, Write reasonable, Reading advanced.
\end{tabular}

%----------------------------------------------------------------------------------------

%----------------------------------------------------------------------------------------
%	Prizes and Titles
%----------------------------------------------------------------------------------------

\section{Titles}

\begin{tabular}{ @{} >{\bfseries}l @{\hspace{5ex}} l }
2013 & \multicolumn{1}{p{15cm}}{\raggedright Silver Medal Olimpíada de Matemática do Cone Sul} \\
2012 & \multicolumn{1}{p{15cm}}{\raggedright Gold Medal na OMCPLP, Olimpíada de Matemática da Comunidade dos Países Língua Portuguesa} \\
2012 & \multicolumn{1}{p{15cm}}{\raggedright Bronze Medal na IYPT Brasil, INTERNATIONAL YOUNG PHYSICISTS' TOURNAMENT} \\
2012 & \multicolumn{1}{p{15cm}}{\raggedright Bronze Medal na OBF, Olimpíada Brasileira de Física} \\
2012 & \multicolumn{1}{p{15cm}}{\raggedright Bronze Medal na OBM, Olimpíada Brasileira de Matemática} \\
2012 & \multicolumn{1}{p{15cm}}{\raggedright Bronze na OQRJ, Olimpíada Brasileira de Química do Estado do Rio de Janeiro} \\
2012 & \multicolumn{1}{p{15cm}}{\raggedright Gold Medal na OMCPLP, Olimpíada de Matemática da Comunidade dos Países de Língua Portuguesa} \\
2012 & \multicolumn{1}{p{15cm}}{\raggedright Silver Medal na IJSO Brasil, International Junior Science Olympiad} \\
2011 & \multicolumn{1}{p{15cm}}{\raggedright Bronze Medal na OMERJ, OMERJ - Olimpíada de Matemática do Estado do Rio de Janeiro} \\
2011 & \multicolumn{1}{p{15cm}}{\raggedright Gold Medal na OBF, Olimpíada Brasileira de Física} \\
2011 & \multicolumn{1}{p{15cm}}{\raggedright Gold Medal na OBM, Olimpíada Brasileira de Matemática} \\
2010 & \multicolumn{1}{p{15cm}}{\raggedright Bronze Medal na OBA, Olimpiada Brasileira de Astronomia} \\
2010 & \multicolumn{1}{p{15cm}}{\raggedright Bronze Medal na OBM, Olimpíada Brasileira de Matemática} \\
2009 & \multicolumn{1}{p{15cm}}{\raggedright Silver Medal na OMERJ, OMERJ - Olimpíada de Matemática do Estado do Rio de Janeiro}
\end{tabular}

%----------------------------------------------------------------------------------------

\section{Productions - Interviews, table, programs e comments midia}

\begin{tabular}{ @{} >{\bfseries}l @{\hspace{5ex}} l }
ROCHA, D. S. & \multicolumn{1}{p{13cm}}{\raggedright A Gold medal nas Olimpíadas de matemática na comunidade dos
países da Língua Portuguesa., 2012} \\
ROCHA, D. S. & \multicolumn{1}{p{13cm}}{\raggedright É possível Estudar Matemática sem sofrer, 2012} \\
ROCHA, D. S. & \multicolumn{1}{p{13cm}}{\raggedright A medalha de ouro nas Olimpíadas de matemática na comunidade dos países da Língua Portuguesa., 2012. (Programa, Programa de Rádio ou TV)} \\
ROCHA, D. S. & \multicolumn{1}{p{13cm}}{\raggedright Reportagem Fantástico. (Programa, Programa de Rádio ou TV)} \\
Link & \multicolumn{1}{p{13cm}}{\raggedright Canal do youtube com entrevistas: https://www.youtube.com/channel/UCvC92Y82h2cBcGgGWBcwBoQ/featured}
\end{tabular}

\end{center}
\end{document}
